% !TEX program = xelatex
\documentclass{beamer}
\usetheme{Singapore}
\usepackage{amsmath}
\usepackage{ctex}
\usepackage{url}
\usepackage{hyperref}

\title{复杂网络中统计规律的检验}
\author{修格致}
\institute{IRSGIS\\ Peking University}
\date{\today}

\begin{document}
    \maketitle    
\begin{frame}{Contents}
    \tableofcontents
\end{frame}
\section{统计检验的基础知识}
\begin{frame}{统计检验}
    \begin{itemize}
        \item 什么是统计检验?
        \item 为什么要做统计检验?
        \item 统计检验的对象是什么?
        \item 怎么算完成了统计检验?
    \end{itemize}
\end{frame}
\begin{frame}{统计检验的常见模型}
    \begin{itemize}
        \item 假设检验
        \item 回归分析
    \end{itemize}
\end{frame}
\begin{frame}{复杂网络中的统计规律}
    \begin{itemize}
        \item 有什么?
        \begin{itemize}
            \item 小世界:三角形的比例[Ref: Nat. Phy. Multiscale Navigation].
            \item 无标度:幂律的度分布$P(k)\sim k^{-\gamma}$.
        \end{itemize}
        \item 有什么问题?
        \begin{itemize}
            \item 数据相关性:规模效应、局部特征.
            \item 统计工具缺陷:幂律数据 v.s. 线性误差.
            \item \dots\dots
        \end{itemize}
    \end{itemize}
\end{frame}
\begin{frame}{文献列表}
    \begin{itemize}
        \item Testing Statistical Laws in Complex Systems
    \item Model Selection and Hypothesis Testing for Large-Scale Network Models with Overlapping Groups
    \item Polya filter
    \end{itemize}
\end{frame}
\section{prl2019, Testing Statistical Laws in Complex Systems}
\begin{frame}{Testing Statistical Laws in Complex Systems}
    \begin{itemize}
        \item 我们该如何正确地检验复杂系统中出现的统计规律?
    \end{itemize}
\end{frame}
\begin{frame}{摘要翻译}
    复杂系统中有着形形色色的统计规律,比如统计单词出现频率的Zipf定律,地震里氏等级的Gutenberg-Richter律,和复杂网络中可能会出现的无标度分布律等。检验这些定律需要更高的统计观点。本letter中,我们讨论了一个复杂系统所生成数据的一个常见现象:这些统计规律是如何受到观测的相关性影响的。我们首先证明了标准的最大似然估计如何得到“第二类错误”(错误的否定了结论)。然后我们提出了一个保守的方法来检验这些规律,并证明了这种方法找到的参数有更小的拒绝率和更大的置信区间。
\end{frame}
\begin{frame}{Trends}
    \begin{itemize}
        \item 1999: BA模型被提出. 随后几年无标度分布在各种网络中被发现. 
        \item 最近五年:很多研究证明,这其中的大部分并不是幂律分布.
        \begin{itemize}
            \item 原因:
            \begin{enumerate}
                \item 数据集变得更大\\(是否从侧面证明:即使数据不是幂律分布的,采样也是幂律分布的?).
                \item 统计方法的提升:从最小二乘拟合变成了最大似然估计. 
            \end{enumerate}
        \end{itemize}
    \end{itemize}
\end{frame}
\begin{frame}{建立正规的统计模型}
    我们检验一个数据集服从何种分布的时候,通常依赖于以下两个假设:
    \begin{enumerate}
        \item 观测数据服从某分布$p(x,\vec{\alpha})$, 比如说幂律分布$$p(x;\alpha)=Cx^{-\alpha}$$
        \item 经验观测$x_1,x_2,\dots,x_N$是独立的, 在这里独立指关于$i$和之前的变量$x_{i-1}$这两个意义.
    \end{enumerate}
    其中第一条讲得是统计定律,后一条是关于统计检验的. 比如说对数似然函数的 加和 依赖于这条独立性. Why?
\end{frame}
\begin{frame}{问题?}
    \begin{itemize}
        \item 复杂网络中的数据相关性与与时间的关系?
        \item 独立性?
    \end{itemize}
    \vspace{1cm}
    规模效应是否与统计检验要求的独立性有着本质的矛盾?
    对假设2的背离可能会导致服从假设1的分布被否定。
\end{frame}



\begin{frame}{Thank You!}
    My personal website: \url{gxiu.github.io}
\end{frame}
\end{document}