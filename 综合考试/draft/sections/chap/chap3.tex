% Copyright (c) 2014,2016,2018 Casper Ti. Vector
% Public domain.

\chapter{动态演化规律}

\section{基于格点的演化博弈模型}

\subsection{Ising模型与传染病的空间传播}

Ising模型原本是Lenz提出来解释铁磁-顺磁体相变的理论模型。Ising模型在统计物理上的意义,是解决了铁磁体到顺磁体的连续相变问题。在一个磁性体系中,如果局部的大部分微粒的磁性朝向都相同,那么整体上,系统也会有相同的磁性朝向;而局部的微粒磁性朝向很均匀的分配的话,那么整个系统就不会对外展现出磁性。而如果对这个系统加上一个外磁场,那么这个体系对外会展现这个外磁场的朝向。也就是,这个体系成为了顺磁体。社会现象中,我们也能找到很多这种性质的系统。这些系统中,也存在着能量的分配,也存在着小组分之间的交互,也找不到什么特征尺度,也存在着那么些个临界状态。本文试图举两个例子来阐述这种性质。这些模型都可以利用Ising模型来解释。我们将考察,人类社会的两种机制(结构、自组织的形成,与熵增过程)的对抗在社会自组织现象中的体现。我们以舆论场为例:在通常状况下,一个区域里的人的意见都是各有不同的。我们以英国的大选为例:将政治讯息当作选举体系中的能量,那么能量是如何在系统中分配的呢?这与Ising模型有着很多的相似之处:每个个体都受两种因素支配:与邻居的交流,和整个舆论场的影响。类比于Ising模型,我们可以问,在舆论场的混乱程度达到什么程度的时候,一点点轻微的舆论动向就可以左右整个选举结果?

\subsection{Rick Durrett的工作}