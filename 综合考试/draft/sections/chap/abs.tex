% Copyright (c) 2014,2016 Casper Ti. Vector
% Public domain.

\begin{cabstract}
	世界各国在快速城市化过程中,集中涌现出很多有共性的模式与问题。其中包括空气污染、城市热岛现象、能源消耗问题、社会空间不平等以及很多与可持续发展有关的问题。因此,对城市结构与演变规律进行建模就显得尤为重要。基于简单规则而可解析的统计物理模型可以在不同的角度给出城市发展机制的阐释和预判。近年来,可供此类研究参考的数据越来越多,建模的置信程度也在逐渐提高。这使得此类简洁有力的物理模型在城市研究中不可或缺。构建可定量研究、能解释尽量多的城市现象、并有预测性的城市科学成为了有潜力的发展方向。本文将总结几类常见的地理现象的物理解释。我们关注的问题有城市人口分布、城市分隔模式、重力与辐射模型、与基于位置的城市功能稳定性模型。我们将给出这些模型的描述与内在机理解释。最后我们将给出原创性的一部分工作,关于城市中亚线性与超线性模式的出现。
	用简单的物理模型来描述复杂的地理现象一直是所有科学家的诉求。本文将总结几类常见的地理现象的物理解释。以图尽量优美地描述地理世界。

	在大数据时代,我们将面对很多不知对错的规律。而\emph{原理}在任何时代都是稀缺的。因果路径可能可以由统计物理的生成方式复现。这也是我们进行这类研究的初衷。

\end{cabstract}

\begin{eabstract}
	The elegant universe.
\end{eabstract}

% vim:ts=4:sw=4
