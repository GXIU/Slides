% Copyright (c) 2008-2009 solvethis
% Copyright (c) 2010-2016,2018 Casper Ti. Vector
% Public domain.
%
% 使用前请先仔细阅读 pkuthss 和 biblatex-caspervector 的文档,
% 特别是其中的 FAQ 部分和用红色强调的部分。
% 两者可在终端/命令提示符中用
%   texdoc pkuthss
%   texdoc biblatex-caspervector
% 调出。

% 采用了自定义的(包括大小写不同于原文件的)字体文件名,
% 并改动 ctex.cfg 等配置文件的用户请自行加入 nofonts 选项;
% 其它用户不用加入 nofonts 选项,加入之后反而会产生错误。
\documentclass[UTF8]{pkuthss}
% 如果的确须要使脚注按页编号的话,可以去掉后面 footmisc 包的注释。
% 注意:在启用此设定的情况下,可能要多编译一次以产生正确的脚注编号。
%\usepackage[perpage]{footmisc}

% 使用 biblatex 排版参考文献,并规定其格式(详见 biblatex-caspervector 的文档)。
% 这里按照西文文献在前,中文文献在后排序(“sorting = ecnyt”);
% 若须按照中文文献在前,西文文献在后排序,请设置“sorting = cenyt”;
% 若须按照引用顺序排序,请设置“sorting = none”。
% 若须在排序中实现更复杂的需求,请参考 biblatex-caspervector 的文档。
\usepackage[backend = biber, style = caspervector, utf8, sorting = ecnyt]{biblatex}

% 对于 linespread 值的计算过程有兴趣的同学可以参考 pkuthss.cls。
\renewcommand*{\bibfont}{\zihao{5}\linespread{1.27}\selectfont}
% 按学校要求设定参考文献列表的段间距。
\setlength{\bibitemsep}{3bp}

% 设定文档的基本信息。
\pkuthssinfo{
	cthesisname = {博士研究生综合考试纸质材料}, ethesisname = {Comprehensive Exams},
	ctitle = {城市中若干问题的生成模型}, etitle = {Modelling Urban Issues},
	cauthor = {修格致},
	eauthor = {Gezhi Xiu},
	studentid = {1801110566},
	date = {\today},
	school = {地空学院},
	cmajor = {地理信息系统}, emajor = {GIS},
	direction = {复杂系统},
	cmentor = {刘瑜\ 教授}, ementor = {Prof.\ Yu Liu},
	ckeywords = {生成模型, 复杂系统, 统计物理, 随机过程}, ekeywords = {Generation models, complex network, statistical physics, stochastic process}
}
% 载入参考文献数据库(注意不要省略“.bib”)。
\addbibresource{thesis.bib}

% 普通用户可删除此段,并相应地删除 chap/*.tex 中的
% “\pkuthssffaq % 中文测试文字。”一行。
\usepackage{color}
\def\pkuthssffaq{%
	\emph{\textcolor{red}{pkuthss 文档模版最常见问题:}}

	\texttt{\string\cite}、\texttt{\string\parencite} %
	和 \texttt{\string\supercite} 三个命令分别产生%
	未格式化的、带方括号的和上标且带方括号的引用标记:%
	\cite{test-en},\parencite{test-zh}、\supercite{test-en, test-zh}。

	若要避免章末空白页,请在调用 pkuthss 文档类时加入 \texttt{openany} 选项。

	如果编译时不出参考文献,
	请参考 \texttt{texdoc pkuthss}“问题及其解决”一章
	“上游宏包可能引起的问题”一节中关于 biber 的说明。%
}

\begin{document}
	% 以下为正文之前的部分,默认不进行章节编号。
	\frontmatter
	% 此后到下一 \pagestyle 命令之前不排版页眉或页脚。
	\pagestyle{empty}
	% 自动生成封面。
	\maketitle
	% 版权声明。封面要求单面打印,故须新开右页。
	\cleardoublepage
	% Copyright (c) 2008-2009 solvethis
% Copyright (c) 2010-2017 Casper Ti. Vector
% All rights reserved.
%
% Redistribution and use in source and binary forms, with or without
% modification, are permitted provided that the following conditions are
% met:
%
% * Redistributions of source code must retain the above copyright notice,
%   this list of conditions and the following disclaimer.
% * Redistributions in binary form must reproduce the above copyright
%   notice, this list of conditions and the following disclaimer in the
%   documentation and/or other materials provided with the distribution.
% * Neither the name of Peking University nor the names of its contributors
%   may be used to endorse or promote products derived from this software
%   without specific prior written permission.
%
% THIS SOFTWARE IS PROVIDED BY THE COPYRIGHT HOLDERS AND CONTRIBUTORS "AS
% IS" AND ANY EXPRESS OR IMPLIED WARRANTIES, INCLUDING, BUT NOT LIMITED TO,
% THE IMPLIED WARRANTIES OF MERCHANTABILITY AND FITNESS FOR A PARTICULAR
% PURPOSE ARE DISCLAIMED. IN NO EVENT SHALL THE COPYRIGHT HOLDER OR
% CONTRIBUTORS BE LIABLE FOR ANY DIRECT, INDIRECT, INCIDENTAL, SPECIAL,
% EXEMPLARY, OR CONSEQUENTIAL DAMAGES (INCLUDING, BUT NOT LIMITED TO,
% PROCUREMENT OF SUBSTITUTE GOODS OR SERVICES; LOSS OF USE, DATA, OR
% PROFITS; OR BUSINESS INTERRUPTION) HOWEVER CAUSED AND ON ANY THEORY OF
% LIABILITY, WHETHER IN CONTRACT, STRICT LIABILITY, OR TORT (INCLUDING
% NEGLIGENCE OR OTHERWISE) ARISING IN ANY WAY OUT OF THE USE OF THIS
% SOFTWARE, EVEN IF ADVISED OF THE POSSIBILITY OF SUCH DAMAGE.

% 此处不用 \specialchap,因为学校要求目录不包括其自己及其之前的内容。
\chapter*{版权声明}
% 综合学校的书面要求及 Word 模版来看,版权声明页不用加页眉、页脚。
\thispagestyle{empty}

任何收存和保管本论文各种版本的单位和个人,
未经本论文作者同意,不得将本论文转借他人,
亦不得随意复制、抄录、拍照或以任何方式传播。
否则一旦引起有碍作者著作权之问题,将可能承担法律责任。

% 若须排版二维码,请将二维码图片重命名为“barcode”,
% 转为合适的图片格式,并放在当前目录下,然后去掉下面 2 行的注释。
%\vfill\noindent
%\includegraphics[height = 5em]{barcode}

% vim:ts=4:sw=4


	% 此后到下一 \pagestyle 命令之前正常排版页眉和页脚。
	\cleardoublepage
	\pagestyle{plain}
	% 重置页码计数器,用大写罗马数字排版此部分页码。
	\setcounter{page}{0}
	\pagenumbering{Roman}
	% 中西文摘要。
	% Copyright (c) 2014,2016 Casper Ti. Vector
% Public domain.

\begin{cabstract}
	世界各国在快速城市化过程中,集中涌现出很多有共性的模式与问题。其中包括空气污染、城市热岛现象、能源消耗问题、社会空间不平等以及很多与可持续发展有关的问题。因此,对城市结构与演变规律进行建模就显得尤为重要。基于简单规则而可解析的统计物理模型可以在不同的角度给出城市发展机制的阐释和预判。近年来,可供此类研究参考的数据越来越多,建模的置信程度也在逐渐提高。这使得此类简洁有力的物理模型在城市研究中不可或缺。构建可定量研究、能解释尽量多的城市现象、并有预测性的城市科学成为了有潜力的发展方向。本文将总结几类常见的地理现象的物理解释。我们关注的问题有城市人口分布、城市分隔模式、重力与辐射模型、与基于位置的城市功能稳定性模型。我们将给出这些模型的描述与内在机理解释。最后我们将给出原创性的一部分工作,关于城市中亚线性与超线性模式的出现。
	用简单的物理模型来描述复杂的地理现象一直是所有科学家的诉求。本文将总结几类常见的地理现象的物理解释。以图尽量优美地描述地理世界。

	在大数据时代,我们将面对很多不知对错的规律。而\emph{原理}在任何时代都是稀缺的。因果路径可能可以由统计物理的生成方式复现。这也是我们进行这类研究的初衷。

\end{cabstract}

\begin{eabstract}
	The elegant universe.
\end{eabstract}

% vim:ts=4:sw=4

	% 自动生成目录。
	\tableofcontents

	% 以下为正文部分,默认要进行章节编号。
	\mainmatter
	% 各章节。
	% Copyright (c) 2014,2016,2018 Casper Ti. Vector
% Public domain.

\chapter{研究背景}

barthelemy的一篇:城市中的统计物理。
% vim:ts=4:sw=4

	% Copyright (c) 2014,2016 Casper Ti. Vector
% Public domain.

\chapter{城市中若干定律的生成模型}

随着人类生存发展需求的提高,城市在信息传递、生产力、政府管理等方面的固有优势使得越来越多的人涌入了城市。这个过程不是简单的,城市规模的变化模式与城市指标的变化并不是简单的线性关系。我们知道丰富的材料在尝试解决这些问题。在这一章中,我们将尝试总结一些城市生长模式的微观解释。

\section{分布定律,增长定律}

在各国的城市研究中存在着一些通用的规律。其中最广为人知的一种是Zipf定律,即城市规模的频数与城市的位序的一个负的方幂成正比。\[P(n)\sim n^{-(1+\beta)}.\]其中$\beta\approx -1$. 根据2010年美国国家统计局的数据在美国最大的几个城市中,纽约的人口约为820万,洛杉矶人口约为390万,芝加哥、休斯顿、费城的人口分别为270万、210万和150万。

\section{复杂网络结点度分布的导出}

我们接下来考察Krapivsky和Redner的网络生长模型,该模型中概括了一般的网络生长度分布的导出规律。

我们考虑直接(GN)、间接(GNR)两种模型的度分布$n_k$,即连结其他结点的数目为某一特定值的结点在网络中所占的比例。GN模型在每个时刻$t$向网络中加入两个结点,它与已有的结点连一条边。我们考察的$N_k(t)=n_kt$,即在某个时刻$t$,网络中度为$k$的结点的数量。我们将模型考虑为时间连续的,根据结点添加规则,我们可以列出生长的微分方程\[\frac{dN_k}{dt} = \frac{(k-1)N_{k-1}-kN_{k}}{2t} +\delta_{k,1}. \] 其中$\delta_{k,1}$给出了度为$1$的模型演化规律。我们将微分方程中的$N_k$由$n_k$替代,可以得到\[n_k = (k-1)n_{k-1}/2-kn_k/2 +\delta_{k,1} \]它依次为\begin{align}
    n_1 = 1-n_1/2, \\ \dots\dots \\ n_k = (k-1)n_{k-1}/2 - kn_{k}/2
\end{align}.
归并同类项,可以得到$n_k = \frac{k-1}{k+2}n_{k-1} = \cdots = \frac{6}{k(k+1)(k+2)} n_1\sim k^{-3}$. 这告诉我们结点的度分布实际上来源于总度数随时间的增加速度。


\section{空间复杂网络的结构特征}

\subsection{核心边缘结构}

航空网络和贸易网络是典型的拥有核心边缘结构的复杂网络\cite{verma2016emergence}。此类网络通常有着很好的嵌套关系和层级结构。此类性质在生态系统中也有着很好的对应。我们在这里介绍层次矩阵和复杂网络两种方法。
	% Copyright (c) 2014,2016,2018 Casper Ti. Vector
% Public domain.

\chapter{动态演化规律}

\section{基于格点的演化博弈模型}

\subsection{Ising模型与传染病的空间传播}

Ising模型原本是Lenz提出来解释铁磁-顺磁体相变的理论模型。Ising模型在统计物理上的意义,是解决了铁磁体到顺磁体的连续相变问题。在一个磁性体系中,如果局部的大部分微粒的磁性朝向都相同,那么整体上,系统也会有相同的磁性朝向;而局部的微粒磁性朝向很均匀的分配的话,那么整个系统就不会对外展现出磁性。而如果对这个系统加上一个外磁场,那么这个体系对外会展现这个外磁场的朝向。也就是,这个体系成为了顺磁体。社会现象中,我们也能找到很多这种性质的系统。这些系统中,也存在着能量的分配,也存在着小组分之间的交互,也找不到什么特征尺度,也存在着那么些个临界状态。本文试图举两个例子来阐述这种性质。这些模型都可以利用Ising模型来解释。我们将考察,人类社会的两种机制(结构、自组织的形成,与熵增过程)的对抗在社会自组织现象中的体现。我们以舆论场为例:在通常状况下,一个区域里的人的意见都是各有不同的。我们以英国的大选为例:将政治讯息当作选举体系中的能量,那么能量是如何在系统中分配的呢?这与Ising模型有着很多的相似之处:每个个体都受两种因素支配:与邻居的交流,和整个舆论场的影响。类比于Ising模型,我们可以问,在舆论场的混乱程度达到什么程度的时候,一点点轻微的舆论动向就可以左右整个选举结果?

\subsection{Rick Durrett的工作}
	\chapter{城市中的标度律}

研究城市中标度律的意义在于,本质上我们可以预测城市在人口变化的时候,城市的各种性质的变化规律。

交通拥堵导致的延迟是一个典型的例子。Barthelemy在2018年PNAS发表了From global scaling to the dynamics of individual cities. 交通拥堵可能不只是人口数的函数。它也与城市的历史进程有关系。基于这样的考虑,我们发现,将不同城市的数据进行混合是有一些问题的。将不同城市的组分进行混合需要依赖很强的同质性假设。这对于很多城市指标都是不成立的。而确定这些同质性假设成立的城市指标也是重要的问题之一。

另外,标度指数也可能是与时间相关的,也可能会收敛到均衡值。但是我们需要更多的实验来完成验证。

充足的城市数据对城市的量化研究有极大的好处。但是上面的例子说明,

	% 正文中的附录部分。
	\appendix
	% 排版参考文献列表。bibintoc 选项使“参考文献”出现在目录中;
	% 如果同时要使参考文献列表参与章节编号,可将“bibintoc”改为“bibnumbered”。
	\printbibliography[heading = bibintoc]
	% 各附录。
	% Copyright (c) 2014,2016 Casper Ti. Vector
% Public domain.

\chapter{附件}
% \pkuthssffaq % 中文测试文字。

% vim:ts=4:sw=4


	% 以下为正文之后的部分,默认不进行章节编号。
	\backmatter
	% 致谢。
	% Copyright (c) 2014,2016 Casper Ti. Vector
% Public domain.

\chapter{致谢}
感谢我的导师刘瑜教授在科研生活中对我的指导。感谢董磊博士在带我进入这个方向的过程中启发我的一切。感谢实验室研究小组每周的讨论。感谢邢潇月同学,使我保持了很多向上的动力。感谢我的父亲修春亮教授,在我成长中不断地启迪。感谢我的母亲柯莹女士,你是最好的妈妈。
% vim:ts=4:sw=4

	% 原创性声明和使用授权说明。
	% Copyright (c) 2008-2009 solvethis
% Copyright (c) 2010-2017 Casper Ti. Vector
% All rights reserved.
%
% Redistribution and use in source and binary forms, with or without
% modification, are permitted provided that the following conditions are
% met:
%
% * Redistributions of source code must retain the above copyright notice,
%   this list of conditions and the following disclaimer.
% * Redistributions in binary form must reproduce the above copyright
%   notice, this list of conditions and the following disclaimer in the
%   documentation and/or other materials provided with the distribution.
% * Neither the name of Peking University nor the names of its contributors
%   may be used to endorse or promote products derived from this software
%   without specific prior written permission.
%
% THIS SOFTWARE IS PROVIDED BY THE COPYRIGHT HOLDERS AND CONTRIBUTORS "AS
% IS" AND ANY EXPRESS OR IMPLIED WARRANTIES, INCLUDING, BUT NOT LIMITED TO,
% THE IMPLIED WARRANTIES OF MERCHANTABILITY AND FITNESS FOR A PARTICULAR
% PURPOSE ARE DISCLAIMED. IN NO EVENT SHALL THE COPYRIGHT HOLDER OR
% CONTRIBUTORS BE LIABLE FOR ANY DIRECT, INDIRECT, INCIDENTAL, SPECIAL,
% EXEMPLARY, OR CONSEQUENTIAL DAMAGES (INCLUDING, BUT NOT LIMITED TO,
% PROCUREMENT OF SUBSTITUTE GOODS OR SERVICES; LOSS OF USE, DATA, OR
% PROFITS; OR BUSINESS INTERRUPTION) HOWEVER CAUSED AND ON ANY THEORY OF
% LIABILITY, WHETHER IN CONTRACT, STRICT LIABILITY, OR TORT (INCLUDING
% NEGLIGENCE OR OTHERWISE) ARISING IN ANY WAY OUT OF THE USE OF THIS
% SOFTWARE, EVEN IF ADVISED OF THE POSSIBILITY OF SUCH DAMAGE.

{
	\ctexset{section = {
		format+ = {\centering}, beforeskip = {40bp}, afterskip = {15bp}
	}}

	% 学校书面要求本页面不要页码,但在给出的 Word 模版中又有页码且编入了目录。
	% 此处以 Word 模版为实际标准进行设定。
	\specialchap{北京大学学位论文原创性声明和使用授权说明}
	\mbox{}\vspace*{-3em}
	\section*{原创性声明}

	本人郑重声明:
	所呈交的学位论文,是本人在导师的指导下,独立进行研究工作所取得的成果。
	除文中已经注明引用的内容外,
	本论文不含任何其他个人或集体已经发表或撰写过的作品或成果。
	对本文的研究做出重要贡献的个人和集体,均已在文中以明确方式标明。
	本声明的法律结果由本人承担。
	\vskip 1em
	\rightline{%
		论文作者签名:\hspace{5em}%
		日期:\hspace{2em}年\hspace{2em}月\hspace{2em}日%
	}

	\section*{%
		学位论文使用授权说明\\[-0.33em]
		\textmd{\zihao{5}(必须装订在提交学校图书馆的印刷本)}%
	}

	本人完全了解北京大学关于收集、保存、使用学位论文的规定,即:
	\begin{itemize}
		\item 按照学校要求提交学位论文的印刷本和电子版本;
		\item 学校有权保存学位论文的印刷本和电子版,
			并提供目录检索与阅览服务,在校园网上提供服务;
		\item 学校可以采用影印、缩印、数字化或其它复制手段保存论文;
		\item 因某种特殊原因须要延迟发布学位论文电子版,
			授权学校在 $\Box$\nobreakspace{}一年 /
			$\Box$\nobreakspace{}两年 /
			$\Box$\nobreakspace{}三年以后在校园网上全文发布。
	\end{itemize}
	\centerline{(保密论文在解密后遵守此规定)}
	\vskip 1em
	\rightline{%
		论文作者签名:\hspace{5em}导师签名:\hspace{5em}%
		日期:\hspace{2em}年\hspace{2em}月\hspace{2em}日%
	}

	% 若须排版二维码,请将二维码图片重命名为“barcode”,
	% 转为合适的图片格式,并放在当前目录下,然后去掉下面 2 行的注释。
	%\vfill\noindent
	%\includegraphics[height = 5em]{barcode}
}

% vim:ts=4:sw=4

\end{document}

% vim:ts=4:sw=4
