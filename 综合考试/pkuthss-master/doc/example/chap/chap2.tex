% Copyright (c) 2014,2016 Casper Ti. Vector
% Public domain.

\chapter{城市中若干定律的生成模型}

随着人类生存发展需求的提高,城市在信息传递、生产力、政府管理等方面的固有优势使得越来越多的人涌入了城市。这个过程不是简单的,城市规模的变化模式与城市指标的变化并不是简单的线性关系。我们知道丰富的材料在尝试解决这些问题。在这一章中,我们将尝试总结一些城市生长模式的微观解释。

\section{分布定律,增长定律}

在各国的城市研究中存在着一些通用的规律。其中最广为人知的一种是Zipf定律,即城市规模的频数与城市的位序的一个负的方幂成正比。\[P(n)\sim n^{-(1+\beta)}.\]其中$\beta\approx -1$. 根据2010年美国国家统计局的数据在美国最大的几个城市中,纽约的人口约为820万,洛杉矶人口约为390万,芝加哥、休斯顿、费城的人口分别为270万、210万和150万。

\section{复杂网络结点度分布的导出}

我们接下来考察Krapivsky和Redner的网络生长模型,该模型中概括了一般的网络生长度分布的导出规律。

我们考虑直接(GN)、间接(GNR)两种模型的度分布$n_k$,即连结其他结点的数目为某一特定值的结点在网络中所占的比例。GN模型在每个时刻$t$向网络中加入两个结点,它与已有的结点连一条边。我们考察的$N_k(t)=n_kt$,即在某个时刻$t$,网络中度为$k$的结点的数量。我们将模型考虑为时间连续的,根据结点添加规则,我们可以列出生长的微分方程\[\frac{dN_k}{dt} = \frac{(k-1)N_{k-1}-kN_{k}}{2t} +\delta_{k,1}. \] 其中$\delta_{k,1}$给出了度为$1$的模型演化规律。我们将微分方程中的$N_k$由$n_k$替代,可以得到\[n_k = (k-1)n_{k-1}/2-kn_k/2 +\delta_{k,1} \]它依次为\begin{align}
    n_1 = 1-n_1/2, \\ \dots\dots \\ n_k = (k-1)n_{k-1}/2 - kn_{k}/2
\end{align}.
归并同类项,可以得到$n_k = \frac{k-1}{k+2}n_{k-1} = \cdots = \frac{6}{k(k+1)(k+2)} n_1\sim k^{-3}$. 这告诉我们结点的度分布实际上来源于总度数随时间的增加速度。


\section{空间复杂网络的结构特征}

\subsection{核心边缘结构}

航空网络和贸易网络是典型的拥有核心边缘结构的复杂网络\cite{verma2016emergence}。此类网络通常有着很好的嵌套关系和层级结构。此类性质在生态系统中也有着很好的对应。我们在这里介绍层次矩阵和复杂网络两种方法。