\chapter{绪论}

纵观全球各地与人类文明建立的各个时期,城市的出现与发展是普遍存在的话题。随着城市化与全球化成为一种必然趋势,预计在本世纪末,世界大多数人口都将生活于城市之中\cite{batty2013}。城市形态各异,却通常有着相似的特征:在城市之间,人口、思想、交易和服务产生了城市间多样的交互流;在城市内部,随着空间的集中,个体之间的距离减小,呈现出特定的空间分布与联系;城市多个组分之间相互作用、互依互存的关系构成了复杂的城市生态,允许技能和能源资源的专业化和共享。如Geddes所言,城市不仅是空间场中的场所,也是时间场中的戏剧\cite{geddes1904}。可以认为,城市是人们聚集在一起相互“互动”的地方,是人类社会发展的必然结果。城市中存在着大量组分的相互作用,使得我们可以观察到的大规模城市结构在时间和空间上的演变\cite{Barthelemy2019}。例如,在经济学中,城市的优点主要被解释为宏观的规模效应的实现,即因规模增大带来的经济效益提高,而城市体系的大小通常用微观的人口数来衡量。这代表了一种通过特定的微观生成机制来研究宏观指标的思想,微观尺度的简单规则会生成复杂的宏观现象。基于这一思路,我们认为构建简洁的物理模型并结合日益丰富的城市数据,可以帮助我们复现城市中的统计属性,理解城市运行机制。

城市现象是多组分相互作用构成的多尺度的复杂现象。从一阶角度来看,以单个城市为研究对象,若这一城市是它邻域上最发达的区域,有着最先进的生产能力,则它可以从周围吸收生产资料,从而更快地促进城市生长\cite{Arbesman2009}。然而,如果将视角扩大到较大区域内的城市体系演化,我们会发现城市之间的增长速度是不尽相同的\cite{BerryThe}。再以解释人口增加导致城市在空间分布特征变化的一般规律为例,人的空间分布表现出从家庭($\sim 0.01$ km)到洲际($\sim 10000$ km)的各种尺度的聚类。通过对经验数据的研究,科学家总结出针对城市规模分布的简单幂律规律,发现了人口密度波动是规模的函数。即\[n(N) \propto N^{−2}.\]这个函数形式的意义在于,规模越大的城市频率越小,并且随城市的位次是一个幂指数为$-2$的幂律衰减形式。在这里,城市的规模主要指的是人口规模。使用随机场理论和统计物理学的技术,我们证明这些幂定律从根本上来说是人类无标度空间集群的结果,也是人类居住在二维表面上的事实。从这个意义上说,两个空间尺度上尺度不变的对称性与城市社会学密切相关。通过构建物理模型,可复现城市中的统计属性,有助于我们深入理解城市分中心的区位分布与规模分布的耦合关系,对多中心性城市的共同特征和面临的城市病出现的临界条件,同时也对我们理解人地关系和城市扩张所面临的瓶颈有着很多启示作用。从二阶联系角度,Castells在1989年将当代城市定义为“流的空间”\cite{castells1989}。场所、空间、与构成它们的活动彼此关联\cite{BerryThe}。对一些综合产业(如电影业)来说,产业的重组与重新聚集,与城市地理中心的模式重建的关联可能会对城市化模式产生重要影响\cite{doi:10.1068/d040305}。人类在不同尺度上的交互使得人类社会在不同层次上呈现出不同的模式。在长期演化过程中,城市尺度逐渐显示出了自己的独特之处。这个尺度是相对固定的,城市内部与城市间的增长规律有显著的区别。在城市间尺度上,经济、人口、基础设施建设等方面的增长速度关于城市规模呈现非线性关系,这导致在大尺度上,城市规模的频率正比于城市排名的一个方幂(Zipf's law)\cite{zipf1949};而在城市内部,人与人的交互、政府对有限的城市资源的配置导致城市空间异质性的出现,形成单中心或者多中心结构。供应能力方面,城市可被视为最小的能承担所有必要的个体需求的单位。移动模式方面,城市内的移动是家-工作地的双锚点结构;城市间呈现单锚点结构。

基于以上观测与问题分析,我们认为,城市科学的研究必须针对各个发展阶段从多尺度、多角度进行研究。从城市发展的各个阶段来看,城市涌现遵循客观的演化规律,城市增长会受到空间资源的限制,在城市空间与资源趋于稳定,城市后期的发展将呈现出多组分的重新配置与复杂的相互作用。在尺度方面,城市可以分为人类分布和活动的个体尺度,城市内部活动和交通的城市内尺度,以及大区域范围的城市间尺度。在多角度方面,我们考虑城市组织的多元构成,各组分间复杂的相互依存、相互促进关系构成了丰富的开放城市系统。人口统计数据库的可访问性和统计物理工具的应用使这个领域的研究受益匪浅,这些工具使人们能够识别和分析城市形态和标度律特征的普遍性。

城市涌现初期,我们关注城市空间特征的统计规律,以及城市生长的演化规律。城市发展速度符合一定的统计规律,由此我们可以挖掘城市空间类型与排序特征,并通过基本的物理模型复现城市现象与秩序\cite{Barthelemy2019}。城市人口的聚集效应实现了城市生长与形态学研究,现有模型重点关注已有人口和资源聚集而形成城市的过程\cite{PhysRevE.58.295,PhysRevLett.112.240601}、以及新的人口和资源如何添加到城市之中的配置规律\cite{ZhangScaling,LiSimple}。通过对一系列生成模型的研究我们发现,保证低阶矩的稳定的同时增高高阶矩是一个得到网络异速增长的有效办法。可以总结为,空间异质性的形成是由局部升矩过程导致的。

在城市内部的进一步发展中,我们将更多地考虑空间与资源环境的限制。城市的发展是长时间尺度下、高度异质性的环境中不断演化的过程,考虑现实资源限制的城市演化模型可以进一步揭示错综复杂的城市现象背后的规律及原理。城市产出(如经济增长,城市创造力等)随城市规模实现超线性的增长\cite{Arbesman2009},而城市基础设施需要在有限资源的限制下覆盖尽量大的区域,最优配置密度的增加与城市规模的增加呈现亚线性关系\cite{PhysRevE.74.016117}。城市中心提供了更多的工作机会,吸引人口的流入,多增加一个中心会使中心区域人口呈现超线性的增加;而随着中心的发展,人口不断流入,中心区域更加拥挤而吸引力降低,其它区域的优势逐渐增加,形成城市多中心,中心个数与人口、总通勤距离与人口都将呈现亚线性关系\cite{fujita1982multiple,ogawa1989nonmonocentric}。区域可达性限制与区域优势的相对大小可用来衡量城市亚线性出现的可能性。

城市有限的环境资源使得城市具有抽象的上限,当城市的空间扩展达到稳定的规模,考虑到更为现实的复杂情形,城市不应仅仅被看作一个简单的多组分实体,而是各组分相互作用的城市生态系统。在前述内容中我们所描述的城市涌现、迁移、收缩等可以看作复杂的城市元素空间相互作用过程的抽象。我们考虑微观元素的空间交互,城市空间超越网络性质的拓扑特征,并基于空间博弈演化的模型,探讨空间纵深与资源调控下城市这一大型系统的稳定性和可持续发展能力。我们将对空间多主体行为,城市多要素的交互关系及城市生态系统的稳定性条件进行探讨,刻画城市生态系统的复杂性、开放性与变动性。