\chapter{绪论}
人类在不同尺度上都存在交互。而不同尺度上的交互使得人类社会在不同层次上呈现出不同的模式。在长期演化过程中,城市尺度逐渐显示出了自己的独特之处。这个尺度是相对固定的,城市内部与城市间的增长规律有显著的区别。在城市间尺度上,城市的在经济、人口、基础设施建设等方面的增长速度关于城市规模呈现非线性关系,这导致在大尺度上,城市规模的频率正比于城市的排名的一个方幂(Zipf's law);而在城市内部,人与人的交互、政府对有限的城市资源的配置导致城市空间异质性的出现,形成单中心或者多中心的模式。

世界(尤其是在新兴国家)快速城市化带来的挑战包括对能源的依赖日益增加,空气污染,社会空间不平等以及环境和可持续性问题。因此,对城市的结构和演变进行建模至关重要,因为决策者需要强有力的理论和新的范式来缓解这些问题。近年来,有关城市系统的可用数据不断增加,这使得我们有可能构建量化的城市科学,以识别和建模城市的种种基本现象。统计物理学通过引入能够沟通理论和实证结果的工具和概念,在这项工作中发挥了重要作用。我们关注城市的一些基本问题:城市人口的分布;偏析现象和自旋模型活动组织的多中心过渡;受重力和辐射概念启发的关于流动性和模型的能量考虑;运输过程中排放的二氧化碳;最后,比例尺描述了城市发展时各种社会经济和基础设施的发展方式。



物理学理解复杂系统的一种方法是,构建一个微观模型,从而得出可以在经验数据上进行检验的大规模预测和分析。 这种思路已经取得了许多成功,单严重依赖于经验观察和普适规律的存在性。近些年来,随着大家的数据意识逐渐增强,地理学在传统研究方式之外增添了很多量化的做法。通过获得有关技术社会系统的大量数据,我们可以开启对社会总体状态的定量分析。 特别是,信息和通信技术已经成为有关城市数据的重要来源,这些新技术也很可能会影响城市动态。

另一方面,城市增长的动态模式在很大程度上是不确定的。
