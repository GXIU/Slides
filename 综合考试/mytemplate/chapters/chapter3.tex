\chapter{空间生态系统模型}

Gardner和Ashby提出,大型复杂系统只能在一定临界程度的连通性上面是稳定的。他们的结论是基于大小为4,7,10的系统的计算机模拟研究。May在自己的工作中通过更多的变量来完善这个研究。找到稳定与不稳定的界限仍然是研究问题的关键。

本实验中,作者采用的记号有:连通率\(C\),物种数\(n\)。

物种之间的关系通常是非线性的。但是这样的系统都可以在均衡点附近做Taylor
expanding。这样的话,这些系统的均衡就可以由这个方程刻画:

\[\frac{dx}{dt} = Ax\]

矩阵\(A\)的对角线元素设置为\(-1\),这是为了考虑进时间因素。每个元素都等概率的是正数或者是负数。然后取各种值的概率都是相等的。我们将他们的均值设置为\(0\),方差的均值设置为\(\alpha\)。方差可以作为平均交互强度的代表。

随机性只在选择物种的时候出现。之后的演化过程中,系统就是完全确定的了。

\begin{itemize}
\item
  系统是稳定的,当且仅当\(A\)的所有特征值都有负的实部。那么交互矩阵是随机的,两个参数满足什么关系的时候系统是稳定的呢?稳定的概率\(P(n,\alpha)\)又是多大呢?

  \begin{itemize}
  \item
    如果\(\alpha\sqrt{n}<1\),系统几乎一定是稳定的;
  \item
    如果\(\alpha\sqrt{n}>1\),系统几乎一定是不稳定的。
  \item
    相变区域的宽度正比于\(n^{-2/3}\)
  \end{itemize}
\item
  连通度\(C\)

  \begin{itemize}
  \item
    定义:非零元素的比例。
  \item
    有了这个量之后,\(\alpha^2C\)就取代了\(\alpha^2\)的角色。系统稳定的临界概率\(\alpha\sqrt{nC}\)。
  \end{itemize}
\end{itemize}

对于生态的话题,这个采样方式有着很普适的作用。

12物种的群落,连通度是15\%,稳定概率为0;但是3个4物种的群落,连通概率是45\%,这样的群落有35\%的概率是稳定的。

四十年前,May证明了,足够大或复杂的生态网络持续存在的可能性接近零,与之前的预期相反。可以分析物种随机相互作用的大型网络。然而,在自然系统中,物种对具有明确定义的相互作用(例如捕食者
-
猎物,共生或竞争)。在这里,我们将May的结果扩展到这些关系,并发现稳定的\textbf{捕食者
-
猎物相互作用}与不稳定的\textbf{共生和竞争}相互作用之间存在显着差异。我们为所有案例提供分析稳定性标准。我们使用这些标准来证明,当一个现实的食物网结构被施加或者存在大量弱互动的优势时,捕食者
-
猎物网络的稳定概率会降低。同样,在二分互惠网络中,稳定性受到嵌套性的负面影响。通过将网络结构和相互作用强度的贡献分离到稳定性,可以发现这些结果。只要捕食者
- 猎物对紧密耦合,稳定的捕食者 -
猎物网络可以是任意大而复杂的。稳定性标准可广泛应用,因为它们适用于任何微分方程组。

\begin{itemize}
\item
  May的定理:

  \begin{itemize}
  \item
    交互矩阵\(M_{S\times S}.\)
    描述了\(j\)对\(i\)在均衡点附近的影响。在他的工作中,对角线的系数是\(-1.\)
    其余的系数以概率为\(C\)从一个确定分布中取,以概率为\(1-C\)取为\(0.\)
    对于这样的矩阵,只要复杂度\(\sigma\sqrt{SC}>1,\)稳定的概率就是\(0.\)
    局部稳定性衡量了一个系统经过扰动后回到均衡的概率。对于不稳定的系统,即使极小的扰动,都会使得系统远离均衡。这潜在的会使得一些物种消失。所以我们基本上不可能见到过于丰富的物种丰富度,或者过于高的连通度。数学上,一个均衡是稳定的,如果交互矩阵所有的特征值的实部都是负的。
  \item
    局部稳定性只是均衡点附近的性质。但是自然系统通常被认为是远离均衡的。但是,基于局部稳定性的方法还是很适合分析聚星系统。这些系统的经验的参数化通常是不适用的。这些方法是通用的,所以这些方法可以用到各种微分方程系统中。
  \item
    May的矩阵有着随机的结构。每一对的交互都有着随机的交互可能。但是,这种随机结构也显示,对于大的\(S\),我们可以将这种随机性翻译成确定的交互频率。所以这种矩阵可以描述交互的很精确的结构。举例来说,捕食-被捕食交互比自养交互要频繁一倍。
  \end{itemize}
\end{itemize}

\section{人群交互模型}

这个部分主要基于几篇文章:Simplicial Models of Social Contation\cite{IacopiniSimplicial}和Spatially Distributed Social Complex Networks\cite{PhysRevX.4.011008}

人类局部交互模型的通常假设是短程交互和长期迁移的组合。每个个体(agent)有一定强度的个人属性,这个属性可能会被其他人影响,进而发生改变。一个例子是人类方言的变迁过程\cite{PhysRevX.7.031008}。

局部交互的方法通常由简单交互过程(基于Ising模型)或者博弈分析(基于纳什均衡)\cite{grimalda2016social,Mussa2019Urbanity} 来得到。

\section{局部交互背景下的注意力流动模型}

\subsection{社群结构的新理解}

近十年来,学界对生成模型的研究不仅仅限于联系的建立,更在于社会网络功能的形成。由于社群结构可以被看作城市生活的基本单元,进而划分空间集聚和空间异化效应的特征尺度,社群结构的定义与性质成为了一个很重要的研究对象。我们重点考察的有两个方面,首先是社会网络的层级结构(Hierarchy),其次是社会网络的社群形成规律。

Hébert-Dufresne等人在2011年的文章\cite{PhysRevLett.107.158702} 中提出了结构偏好依附模型(Structural Preferential Attachment)。该模型可以生成复杂网络的很多性质。比如尺度无关性、模度性和自相似性。并且将这些性质统一在不以连边为基础的的无标度网络系统中。这带来了一个新的观测复杂网络的视角:社区(在结点和边之外的视角)。这个模型可以通过预测其社群结构来复现社会/信息网络。更重要的是,结点和社群是如何连通的。而这通常是一个自相似的结构。

我们首先来理解如何将空间对象抽象成网络。网络可以看作是空间对象的同质个体之间的关联模式。进而一个定义良好的网络抽象应该自然而然是无标度的,因为抽象尺度应该与问题本身是无关的。这个理念的正确性保证了\emph{采样}在复杂网络研究中的有效性。与此相关的是\cite{PhysRevLett.107.158702}中的一个重要观念:将复杂系统简化为最简单的形式,同时保留其重要属性,有助于独立于其性质对行为进行建模。我们认为,抽象出来的网络可以构建对不同是一个普适类。偏好依附(preferential attachment)就是一个重要的普适类。这个机制可以解释各个领域中富者愈富的情形。

Some features of the spread of epidemics and information on a random graph

随机图是社会和技术网络的有用模型。 迄今为止,该领域的大多数研究都涉及图形的几何特性。 在这里,我们关注网络上发生的过程。 我们特别感兴趣的是,它们在网络上的行为与均匀混合种群或在生态模型中常用的规则格上的行为有何不同。

\section{社会网络的层级结构}
