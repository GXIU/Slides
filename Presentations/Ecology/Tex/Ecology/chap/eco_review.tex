\begin{frame}{To maintain biodiversity}
    \begin{itemize}
<<<<<<< HEAD:Presentations/Ecology/Tex/Ecology/chap/eco_review.tex
        \item What what?
=======
    	\item 
>>>>>>> 2b0938a3f53ac99e5359efda11db924844637fd5:Presentations/Ecology/Tex/chap/eco_review.tex
    \end{itemize}
\end{frame}

\begin{frame}{Spatial Structures of a System}
	\begin{itemize}
		\item The spatial structure of a system constrains 
		\begin{itemize}
			\item who interacts with whom (interaction partner) 
			\item who acquires new traits from whom (role model)
		\end{itemize}
		\item a spatial structure promotes cooperation (spatial reciprocity) when interaction partners overlap role models.
		\item strong social ties might hinder, while asymmetric spatial structures for interaction and trait dispersal could promote cooperation.
	\end{itemize}
 An analytical formula to predict when natural selection favors cooperation where the effects of a spatial structure are described by a single parameter
\end{frame}

\begin{frame}{公共物博弈}
	包含$n$个主体的公共物博弈(PGG)
\end{frame}