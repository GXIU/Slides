%------------------------------------------------
\subsection{研究现状}

\begin{frame}{Allee 效应的形成原因}
    \begin{itemize}
        \item cubic model: 来自 Volterra 的工作,三次多项式作为 $f$ 的形式。它的原因是:两种性别之间的产生交互的频率正比于 $\rho^2$. 形式为:\[frac{d\rho(t)}{dt} = -a_1\rho + +(a_2-a_3\rho)\rho^2 = -a_1\rho + a_2\rho^2 - a_3\rho^3.\]
        \item 少伴侣的逻辑斯蒂找到伴侣的概率\[\mathcal{P} = \frac{\rho}{\rho +\theta}.\]其中$\theta$是人口中以$1/2$概率找到伴侣的人口数。显然这是一个可以找得到Allee效应的方式。
        \item \dots
    \end{itemize}
\end{frame}

\begin{frame}{研究现状:与扩散和流行病学的关系}
扩散过程
    \begin{itemize}
        \item 多种基因型的共存
        \item 扩张的节律
        \item ……
    \end{itemize}
增加对流行病学的理解
\begin{itemize}
    \item 最小人口单元问题
    \item 改变局部密度对流行病的影响
\end{itemize}
\end{frame}

\begin{frame}{流行病的时空传播规律}
    时空传播规律与医疗系统设计和对新疫情出现时的理解息息相关。流行病的具体传播规律,不光是参数的设定,更是大类模型的选择。要关注非线性、时空异质性、非马氏性等重要因素。
    \begin{itemize}
        \item Travelling waves and spatial hierarchies in measles epidemics, Bryan Grenfell et. al., Nature 2001
        \item Power laws governing epidemics in isolated populations, C. J. Rhodes and R. M. Anderson
        \begin{itemize}
            \item 与稀树草原-森林演化的林火模型之间的关系
        \end{itemize}
        \item Synchrony, Waves, and Spatial Hierarchies in the Spread of Influenza, Science 2006
        \begin{itemize}
            \item 小波相位分析,可以实现动态非平稳性的探测
        \end{itemize}
        \item From individuals to epidemics
    \end{itemize}
\end{frame}

\begin{frame}{流行病的时空传播规律}
    Travelling waves and spatial hierarchies in measles epidemics, Bryan Grenfell et. al., Nature 2001
    \begin{itemize}
        \item 考虑两个相关的城镇,都有着双年的麻疹周期。进一步假设一个城市够大,超过CCS(大约是300,000),另一个城市规模比CCS小。我们有:
        \begin{itemize}
            \item 在一次大型流行病之后,S的密度构成了一个确定性的阈值,在阈值之上的话,另一场大流行就会到来。
            \item 在大城镇中,麻疹依然会地区性地不时出现,所以只要有效$R_0$在某一瞬间超过$1$时,疫情就会马上爆发。这个阈值与适龄儿童的积累有关系,并且用季节性传染率来修正。
            \item 小城市:在epidemic后$I$在局部消失,所以,另一个大流行只会在接收到infective spark的时候出现,通常来自于更大的城镇。
        \end{itemize}
        \end{itemize}
\end{frame}

\begin{frame}{模型}
    \begin{itemize}
        \item 作者给出了下面三个假设:
        \begin{itemize}
            \item 大小城镇的空间镶嵌(spatial mosaic)对疾病的时滞有影响,如果有一组小城镇在大城市周围,那就会生成层次性的流行病行波。
            \item 作为推论,大城市的组合(都在CCS之上),就会生成高度同步的流行病。因为非线性相会锁住季节性增强的震荡子。
            \item 最后,弱耦合的/距离较远的中心应该有更强的倾向来进入相反的双年吸引子。
        \end{itemize}
        % \item 对偶映射格子,在时刻$n+1$, 流行病的强度$\lambda_{n+1}$又下式给定
        % $$
        % \lambda_{n+1} = \beta_n S_n (I_n + \theta_n)^\alpha,
        % $$
        % 其中,$\beta_n$是季节传染率,$\alpha$是混合率,$\theta_n$是(随机)接触率,假设是泊松分布$\text{Poi}(m\bar{I}_n).$的,$m$是空间耦合率,$\bar{I_n}$是上一个时刻邻居们染病的数量。这样的话,疾病生灭过程就由下面的负二项分布刻画:
        % $$
        % I_{n+1} \sim NegBin(\lambda_{n+1},I_n).
        % $$
        % 健康人的平衡方程就是$S_{n+1} = S_n +B-I_{n+1}$.             
    \end{itemize}
\end{frame}

\begin{frame}{生物入侵}
    \begin{itemize}
        \item Johnson, D., Liebhold, A., Tobin, P. et al. Allee effects and pulsed invasion by the gypsy moth. Nature 444, 361–363 (2006)
        \begin{itemize}
            \item periodic pulsed invasions
            \item how an interaction between strong Allee effects (negative population growth at low densities) and stratified diffusion (most individuals disperse locally, but a few seed new colonies by long-range movement) can explain the invasion pulses
            \item Adult female gypsy moths are flightless, and ballooning of 1st instars usually occurs only over short distances. This leads to stratified diffusion. 
            \item Strong Allee effects can influence species ranges by prohibiting range expansion, termed `invasion pinning'.
        \end{itemize}
        \item Impacts of preference and geography on epidemic spreading, Xin-Jian Xu, et. al. Phys. Rev. E 76, 056109 2007
        \item Epidemic thresholds in dynamic contact networks, J. R. Soc. Interface (2009) 6, 233–241, Erik Volz, and Lauren Ancel Meyers
    \end{itemize}
\end{frame}
%------------------------------------------------
\subsection{问题归纳}

\begin{frame}{问题归纳}
    \begin{enumerate}
        \item Allee Threshold 对于城市的意义:
        \begin{itemize}
            \item 临界密切接触人口单元
        \end{itemize}
        \item 移动性限制与疾病最优阈值
    \end{enumerate}
\end{frame}