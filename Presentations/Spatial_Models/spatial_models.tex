\documentclass{beamer}
\usetheme{Singapore}
\usepackage{bookmark}
\usepackage{cite}
\usepackage{graphicx}
\usepackage{amsmath}
\usepackage{hyperref}
\usepackage{graphicx}% Include figure files
\usepackage{dcolumn}% Align table columns on decimal point
\usepackage{subfigure}
\usepackage{float}
\usepackage{ctex}
\usepackage{url}
\usepackage{bm}% bold math
\usepackage{hyperref}% add hypertext capabilities
\usepackage[mathlines]{lineno}% Enable numbering of text and display math

\title{Spatial Models \& TOC}
\author{修格致}
\date{\today}
\institute{IRSGIS\\Peking U}

\begin{document}
    
\maketitle

\begin{frame}{Table of Contents (also TOC)}
    \tableofcontents    
\end{frame}

\section{About TOC}

\begin{frame}{The tragedy of the commons}
    \begin{quote}{Garrett Hardin:}
        \vspace{0.5cm}
\\        The population problem has no technical solution;\\
        it requires a fundamental extensioin in morality.
    \end{quote}
\end{frame}

\begin{frame}{The tragedy of the commons}
    A tragedy of the commons (TOC) occurs when individuals acting in their own self-interest deplete commonly held resources, leading to a worse outcome than had they cooperated.
\end{frame}

\begin{frame}{Reputation helps solve the 'TOC'}
    Nature 2002
    如果每个人都可以任意过度使用资源,那么在很多的社会困局中,TOC都会涌现。公共资源实验的结果,通常是群体受益的情形不会出现。由于个体和国家都会出现在多个社会博弈中,这些社会博弈的交互通常会产生复杂的结果,这些结果往往保存了公共资源。\textbf{非直接互惠}就是基于名声的一种策略,它可以高度保存合作意愿。本文证明了改变公共物博弈和非直接互惠的轮次可以对保存公共物有着很高的价值。但是如果非直接互惠不存在,对公共物的保护行为很快就消失了。改变博弈给了所有人更高的收益。在很多社会博弈中,名声可以作为一种通货。我们的方法可以验证很多社会问题的可解性。
\end{frame}

\begin{frame}{Current frameworks}
    \begin{itemize}
        \item Evolutionary dynamics arising from a TOC dilemma can be modeled in terms of changes in the frequencies of individuals from two populations, cooperaters and defecters. 
        \item  Individuals interact and receive payoffs that depend on their strategy and the strategy of their opponent, where payoff can be modeled by the payoff matrix,$$A = \begin{Bmatrix}
            R &S \\T &P
        \end{Bmatrix}$$ representing the system's fitness.
        \item The outcome of TOC is measured by the frequency of co-operators and defectors $(x, 1-x)$, and the resources.
        \item This framework is not a zero-sum game.
    \end{itemize}
\end{frame}

\begin{frame}{Current frameworks -- equations \& conditions \small{PhysRevLett.122.148102}}
    \begin{itemize}
        \item fitness \begin{equation}
            \dot{x} = x(1-x)[r_c(x,A)-r_D(x,A)]
        \end{equation}
        $r_C,r_D$ : context-dependent fitness
        payoff to cooperators and defectors, respectively.
        
        \begin{itemize}
            \item TOC's occurrence condition: $T>R>P>S$.
        \end{itemize}
        \item To address the reproductive case: resource-dependent payoff matrices $$A(n)=A_0(1-n)+A_1(n),$$ where $n\in[0,1].$

    \end{itemize}
\end{frame}

\section{Individual-based coevolutionary game}
\begin{frame}{Individual-based coevolutionary game}
    \begin{itemize}
        \item Intuitions on the emergent dynamics of social context and resources:
        \begin{enumerate}
            \item to assess the influence of noise
            \item spatially explicit interactions
        \end{enumerate}
        \item Schemes:
    \end{itemize}
\end{frame}

\begin{frame}{Individual-based coevolutionary game}
    \begin{itemize}
        \item Results
        \begin{itemize}
            \item Transition rate for \#C and \#D. Furthermore, the limiting frequency of cooperaters $\lim_{N,n_c\rightarrow \infty}\frac{n_C}{N} $
        \end{itemize}
        \item Problems: is such frequency convergent or divergent?
        \begin{itemize}
            \item Recalling a Cauchy distribution, or a Lorenz oscillator.
            \item In other words, is the society ending up in tragedy?
        \end{itemize}
    \end{itemize}
\end{frame}

\begin{frame}{Individual-based coevolutionary game}

\end{frame}

\begin{frame}{Demongraphic noise and spatial structure}

\end{frame}

\section{Coexistence in cities}

\begin{frame}{Coexistence in cities}
    \begin{itemize}
        \item \emph{City} is a concrete of aggregate effect. Such complex is not only the sum of different parts, but also the chemistry through each part.
        \vspace{0.5cm}
        \item The resource of cities?
        \begin{itemize}
            \item Citizens, which are usually somewhat \emph{evenly} distributed spatially.
            \item Firms, diverse in reliance on amongst distances.
        \end{itemize}
        \item The costs of cities?
        \begin{itemize}
            \item Dynamics of input and output.
        \end{itemize}
    \end{itemize}
\end{frame}

\begin{frame}{Establishing models}
    \begin{itemize}
        \item Task: Predicting the emergence of new firms over a city's space.
        \item[0.] Collect financial statements of companies and the decay of communication distances, establish the IOs of every trade and the current cross matrix;
        \item[1.] Vectorize the factors of different firms by reliance on differnet trades;
        \item[2.] Establish a dynamical matrix of size $kN\times kN$, new companies may emerge at some optimal location to take charge for urban development;
        \item[3.] simulate until it ends or a sufficiently long time;
        \item[4.] Evaluate the diversity of trades and the fitness of industrial structures.
    \end{itemize}
\end{frame}

\begin{frame}{Expecting}
    \begin{itemize}
        \item Some cities' industrial structures may lead to TOC: companies harm their city when pursuiting their own benefits.
        \item Diversity of cities diverges according to the initial conditions since both the specialization and diversification exist.
        \item Emergence of new companies‘ spatial and size distribution.
    \end{itemize}
\end{frame}

\section{References}
\begin{frame}{References}
    \begin{itemize}
        \item Spatial Interactions and Oscillatory Tragedies of the Commons
        Yu-Hui Lin and Joshua S. Weitz
        Phys. Rev. Lett. 122, 148102 – Published 12 April 2019
        \item Evolution of cooperation in stochastic games, Christian Hilbe, Štěpán Šimsa, Krishnendu Chatterjee \& Martin A. Nowak 
        Nature - Published: 04 July 2018
        \item Reputation helps solve the ‘tragedy of the commons’
        Manfred Milinski, Dirk Semmann \& Hans-Jürgen Krambeck Nature - Published: 24 January 2002
        \item Cooperating with the future
        Oliver P. Hauser, David G. Rand, Alexander Peysakhovich \& Martin A. Nowak Nature Published: 25 June 2014
        \item Evolutionary games and spatial chaos
        Martin A. Nowak \& Robert M. May Published: 29 October 1992
    \end{itemize}
\end{frame}
\begin{frame}{reference}
    \bibliography{ref}
\end{frame}
\end{document}