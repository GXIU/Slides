\documentclass{beamer}
\usepackage[utf8]{ctex}
\usetheme{Singapore}
% \usetheme[euler=false,titlepagebg=B]{ut}
% \usetheme[titlepage=C,debug]{ut}
% \usetheme[titlepage=A]{ut}


\title[Hierarchical Mobility]{Hierarchical organization of urban mobility and its connection with city livability}
% \subtitle[Short subtitle]{I am not using any subtitles}
\author[G. Xiu]{Gezhi Xiu} %  
\institute[IRSGIS PKU]{Complexity Research Group,\\Peking University}
\date[\today]{\today}

\begin{document}


\maketitle
\begin{frame} \frametitle{Abstract}

\begin{itemize}
\item extract global Intra-urban trips
\item develop a metric
  \begin{itemize}
    \item classify cities
    \item establish a connection between mobility organization and key urban indicators
  \end{itemize}
\end{itemize}
Cities with strong hierarchical mobility structure display an extensive use of public transport, higher levels of walkability, lower pollutant emissions per capita and better health indicators.
\end{frame}

\begin{frame}[fragile] \frametitle{Background}
\begin{itemize}
  \item the increase of population,along with the congestion induced by the concentration ofactivity, drives cities from a monocentric to a polycentricconfiguration
  \item urban sprawl
  \begin{itemize}
    \item compact
    \item centers far apart
  \end{itemize}
  \item indicators of urban organization
\end{itemize}
\end{frame}

\begin{frame}[fragile] \frametitle{methods}
\begin{itemize}
\item ML segment a raw GPS trace intosemantic trips
\item The tripflow data can be interpreted as weighted networksformed by S2 cells as nodes, andflows as link weights.
\end{itemize}

\end{frame}

\begin{frame}[fragile] \frametitle{\textbackslash utbeamerset}}
\\utbeamerset\{optiona=valuea, optionb=valueb, optionc=valuec\}
\end{semiverbatim}
}
\end{frame}

\begin{frame}[fragile] \frametitle{Layout of the Title Page}
\begin{itemize}
\item The title page consists of three boxes, A, B and C. Box A contains the title and subtitle. Box B contains the author names, affiliations and date. Box C contains the titlegraphic. The boxes are positioned absolutely on the page. The position and the size of the boxes can be adjusted with the \verb!\utbeamerset! command. Box A has top left coordinate (\verb!tpboxax!,\verb!tpboxay!), width \verb!tpboxawd! and height \verb!tpboxaht!. All sizes and positions are in mm.

\item Overview of all options that can be used in \verb!\utbeamerset!:
tpboxax, tpboxay, tpboxawd, tpboxaht, tpboxbx, tpboxby, tpboxbwd, tpboxbht, tpboxcx, tpboxcy, tpboxcwd, tpboxcht.

\item By default the boxes themselves are not visible. By using the option \verb!debug (\usetheme[debug]{ut})! the boxes are visible, faciliting the layout.
\end{itemize}
\end{frame}


%%%%%%%%%%%%%%%%%%%%%%%%%%%%%%%%%%%%%%%%%%%%%%%%%%%%%%%%%%%%%%%%%%%%%%%%%%%%%%%%%%%
%%%%%%%%%%%%%%%%%%%%%%%%%%%%%%%%%%%%%%%%%%%%%%%%%%%%%%%%%%%%%%%%%%%%%%%%%%%%%%%%%%%
%%%%%
%%%%%
%%%%%
%%%%%%%%%%%%%%%%%%%%%%%%%%%%%%%%%%%%%%%%%%%%%%%%%%%%%%%%%%%%%%%%%%%%%%%%%%%%%%%%%%%
%%%%%%%%%%%%%%%%%%%%%%%%%%%%%%%%%%%%%%%%%%%%%%%%%%%%%%%%%%%%%%%%%%%%%%%%%%%%%%%%%%%
\begin{frame} \frametitle{Fonts}
Use xelatex to get Arial fonts.
\end{frame}

%%%%%%%%%%%%%%%%%%%%%%%%%%%%%%%%%%%%%%%%%%%%%%%%%%%%%%%%%%%%%%%%%%%%%%%%%%%%%%%%%%%
%%%%%%%%%%%%%%%%%%%%%%%%%%%%%%%%%%%%%%%%%%%%%%%%%%%%%%%%%%%%%%%%%%%%%%%%%%%%%%%%%%%
%%%%%
%%%%%
%%%%%
%%%%%%%%%%%%%%%%%%%%%%%%%%%%%%%%%%%%%%%%%%%%%%%%%%%%%%%%%%%%%%%%%%%%%%%%%%%%%%%%%%%
%%%%%%%%%%%%%%%%%%%%%%%%%%%%%%%%%%%%%%%%%%%%%%%%%%%%%%%%%%%%%%%%%%%%%%%%%%%%%%%%%%%
\begin{frame} \frametitle{Feedback}
Feedback is greatly appreciated. \\

Jasper Goseling (j.goseling@utwente.nl)
\end{frame}

\appendix

% \begin{chapterframe}
% \Large Introduction to the Next Part
% \end{chapterframe}

\begin{frame} \frametitle{A Normal Frame}
A list
\begin{itemize}
\item First item
\item Second item
\end{itemize}

\begin{block}{A Block}
Content of the block.
\end{block}
\end{frame}

\begin{frame} \frametitle{Testing the Math Font}
\begin{equation*}
\alpha, \beta, \gamma, 1, 2, 3, \int_a^b f(x)dx, \sum_{i=1}^N, \mathbb{E}[X].
\end{equation*}
\end{frame}
\end{document}

