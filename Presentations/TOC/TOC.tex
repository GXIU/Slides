\documentclass{beamer}
\usetheme{Singapore}
\usepackage{bookmark}
\usepackage{cite}
\usepackage{graphicx}
\usepackage{amsmath}
\usepackage{hyperref}

% \bibliographystyle{plain}
% \bibliography{ref}

\title{Spatial Divergence \& Oscillatory TOCs}
\author{Gezhi Xiu}
\date{\today}
\institute{IRSGIS\\Peking U}

\begin{document}
    
\maketitle

\begin{frame}{Table of Contents (also TOC)}
    \tableofcontents    
\end{frame}

\section{About TOC}

\begin{frame}{The tragedy of the commons}
    \begin{quote}{Garrett Hardin:}
        \vspace{0.5cm}
\\        The population problem has no technical solution;\\
        it requires a fundamental extensioin in morality.
    \end{quote}
\end{frame}

\begin{frame}{The tragedy of the commons}
    A tragedy of the commons (TOC) occurs when individuals acting in their own self-interest deplete commonly held resources, leading to a worse outcome than had they cooperated.
\end{frame}

\begin{frame}{Keys to TOC}
    \begin{itemize}
        \item Macro-scale: 
        \begin{itemize}
            \item game
            \item environment
        \end{itemize}
        \item Individual level:
        \begin{itemize}
            \item divergence of incentives \& pay-offs
        \end{itemize}
    \end{itemize}
\end{frame}

\begin{frame}{Current frameworks}
    \begin{itemize}
        \item Evolutionary dynamics arising from a TOC dilemma can be modeled in terms of changes in the frequencies of individuals from two populations, cooperaters and defecters. 
        \item  Individuals interact and receive payoffs that depend on their strategy and the strategy of their opponent, where payoff can be modeled by the payoff matrix,$$A = \begin{Bmatrix}
            R &S \\T &P
        \end{Bmatrix}$$ representing the system's fitness.
        \item The outcome of TOC is measured by the frequency of co-operators and defectors $(x, 1-x)$, and the resources.
        \item This framework is not a zero-sum game.
    \end{itemize}
\end{frame}

\begin{frame}{Current frameworks -- equations \& conditions \small{PhysRevLett.122.148102}}
    \begin{itemize}
        \item fitness \begin{equation}
            \dot{x} = x(1-x)[r_c(x,A)-r_D(x,A)]
        \end{equation}
        $r_C,r_D$ : context-dependent fitness
        payoff to cooperators and defectors, respectively.
        
        \begin{itemize}
            \item TOC's occurrence condition: $T>R>P>S$.
        \end{itemize}
        \item To address the reproductive case: resource-dependent payoff matrices $$A(n)=A_0(1-n)+A_1(n),$$ where $n\in[0,1].$

    \end{itemize}
\end{frame}

\section{Individual-based coevolutionary game}
\begin{frame}{Individual-based coevolutionary game}
    \begin{itemize}
        \item Intuitions on the emergent dynamics of social context and resources:
        \begin{enumerate}
            \item to assess the influence of noise
            \item spatially explicit interactions
        \end{enumerate}
        \item Schemes:
        \begin{figure}
            \centering
            \includegraphics[height = 3cm]{scheme.PNG}
            \caption{Transitions of cooperaters and defectors.}
        \end{figure}
    \end{itemize}
\end{frame}

\begin{frame}{Individual-based coevolutionary game}
    \begin{itemize}
        \item Results
        \begin{itemize}
            \item Transition rate for \#C and \#D. Furthermore, the limiting frequency of cooperaters $\lim_{N,n_c\rightarrow \infty}\frac{n_C}{N} $
        \end{itemize}
        \item Problems: is such frequency convergent or divergent?
        \begin{itemize}
            \item Recalling a Cauchy distribution, or a Lorenz oscillator.
            \item In other words, is the society ending up in tragedy?
        \end{itemize}
    \end{itemize}
\end{frame}

\begin{frame}{references}
    \begin{itemize}
        \item Spatial Interactions and Oscillatory Tragedies of the Commons
        Yu-Hui Lin and Joshua S. Weitz
        Phys. Rev. Lett. 122, 148102 – Published 12 April 2019
    \end{itemize}
\end{frame}
\end{document}